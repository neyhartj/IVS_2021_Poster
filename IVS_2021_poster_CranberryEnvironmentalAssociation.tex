%%%%%%%%%%%%%%%%%%%%%%%%%%%%%%%%%%%%%%%%%
% Jacobs Landscape Poster
% LaTeX Template
% Version 1.1 (14/06/14)
%
% Created by:
% Computational Physics and Biophysics Group, Jacobs University
% https://teamwork.jacobs-university.de:8443/confluence/display/CoPandBiG/LaTeX+Poster
%
% Further modified by:
% Nathaniel Johnston (nathaniel@njohnston.ca)
%
% This template has been downloaded from:
% http://www.LaTeXTemplates.com
%
% License:
% CC BY-NC-SA 3.0 (http://creativecommons.org/licenses/by-nc-sa/3.0/)
%
%%%%%%%%%%%%%%%%%%%%%%%%%%%%%%%%%%%%%%%%%

%----------------------------------------------------------------------------------------
%	PACKAGES AND OTHER DOCUMENT CONFIGURATIONS
%----------------------------------------------------------------------------------------

\documentclass[final]{beamer}

% Use the beamerposter package for laying out the poster
\usepackage[scale=1.24,orientation=portrait]{beamerposter}
% \usepackage[scale=1.24,orientation=landscape]{beamerposter}

\usetheme{confposter} % Use the confposter theme supplied with this template

\usepackage{wrapfig} % Use for wrapping figures

\usepackage[super,numbers]{natbib} % Use for fine-tunning the captions


%% Define the paths to the Mendeley library
\newcommand\mendlibrary{C:/Users/jeffrey.neyhart/Documents/Literature/MendeleyLibrary/library}



%% Configure code blocks
\usepackage{listings} % Use for code blocks
\usepackage{color}

\definecolor{dkgreen}{rgb}{0,0.6,0}
\definecolor{gray}{rgb}{0.5,0.5,0.5}
\definecolor{mauve}{rgb}{0.58,0,0.82}

\lstset{frame=tb,
  language=Java,
  aboveskip=3mm,
  belowskip=3mm,
  showstringspaces=false,
  columns=flexible,
  basicstyle={\small\ttfamily},
  numbers=none,
  numberstyle=\tiny\color{gray},
  keywordstyle=\color{blue},
  commentstyle=\color{dkgreen},
  stringstyle=\color{mauve},
  breaklines=true,
  breakatwhitespace=true,
  tabsize=3
}

% \usepackage{biblatex} % Bibliography

\setbeamercolor{block title}{fg=greenset1,bg=white} % Colors of the block titles
\setbeamercolor{block body}{fg=black,bg=white} % Colors of the body of blocks
\setbeamercolor{block alerted title}{fg=white,bg=blueset1!70} % Colors of the highlighted block titles
\setbeamercolor{block alerted body}{fg=black,bg=blueset1!10} % Colors of the body of highlighted blocks

% Many more colors are available for use in beamerthemeconfposter.sty

%-----------------------------------------------------------

% Define the column widths and overall poster size
% To set effective sepwid, onecolwid and twocolwid values, first choose how many columns you want and how much separation you want between columns
% In this template, the separation width chosen is 0.02 of the paper width and a 3-column layout
% onecolwid should therefore be (1-(ncol + 1)*sepwid)/ncol e.g. (1-(3+1)*0.024)/3 = 0.3013333
% Set twocolwid to be (2*onecolwid)+sepwid = 0.6266666
% Set threecolwid to be (3*onecolwid)+2*sepwid = 0.9279999

\newlength{\sepwid}
\newlength{\onecolwid}
\newlength{\twocolwid}
\newlength{\threecolwid}

%% Poster size
\setlength{\paperwidth}{44in} % A0 width: 46.8in
\setlength{\paperheight}{44in} % A0 height: 33.1in


\setlength{\sepwid}{0.024\paperwidth} % Separation width (white space) between columns
\setlength{\onecolwid}{0.3013333\paperwidth} % Width of one column
\setlength{\twocolwid}{0.6266666\paperwidth} % Width of two columns
\setlength{\threecolwid}{0.9279999\paperwidth} % Width of three columns
\setlength{\topmargin}{-0.5in} % Reduce the top margin size
%-----------------------------------------------------------

\usepackage{graphicx}  % Required for including images

\usepackage{booktabs} % Top and bottom rules for tables

%----------------------------------------------------------------------------------------
%	TITLE SECTION
%----------------------------------------------------------------------------------------

\title{Identification of Candidate Abiotic Stress Tolerance Loci in Wild Cranberry Using Environmental Association} % Poster title
% Add \\ to induce new line

\vspace{1cm}

\author{Jeffrey L. Neyhart\textsuperscript{1*}, Michael Kantar\textsuperscript{2}, Joseph Kawash\textsuperscript{1}, James Polashock\textsuperscript{1}, and Nicholi Vorsa\textsuperscript{3}} % Author(s)

\institute{
\textsuperscript{1}USDA-ARS, Genetic Improvement for Fruits and Vegetables Laboratory, \textsuperscript{2}Department of Tropical Plant and Soil Sciences, University of Hawai'i at Mānoa, and \textsuperscript{3}Department of Plant Biology, Rutgers University}

% *Contact: Email - \href{mailto:jeffrey.neyhart@usda.gov}{jeffrey.neyhart@usda.gov}


%----------------------------------------------------------------------------------------

\begin{document}

\addtobeamertemplate{block end}{}{\vspace*{2ex}} % White space under blocks
\addtobeamertemplate{block alerted end}{}{\vspace*{2ex}} % White space under highlighted (alert) blocks

\setlength{\belowcaptionskip}{2ex} % White space under figures
\setlength\belowdisplayshortskip{2ex} % White space under equations

\begin{frame}[t] % The whole poster is enclosed in one beamer frame

\begin{columns}[t] % The whole poster consists of three major columns, the second of which is split into two columns twice
% the [t] option aligns each column's content to the top

%----------------------------------------------------------------------------------------

\begin{column}{\sepwid}\end{column} % Empty spacer column

\begin{column}{\onecolwid} % The first column


%----------------------------------------------------------------------------------------
%	TAKEAWAYS
%----------------------------------------------------------------------------------------

\setbeamercolor{block alerted title}{fg=white,bg=orangeset1!70} % Change the alert block title colors
\setbeamercolor{block alerted body}{fg=black,bg=orangeset1!10} % Change the alert block body colors

\begin{alertblock}{\Large{Takeaways}}

% \begin{large}
\begin{itemize}
  \item \textbf{Environmental association may help identify abiotic stress-adaptive genetic variation in wild cranberry}
  \vspace{0.5cm}
  \item \textbf{We detected 143 associations between marker loci and environmental variables}
  \vspace{0.5cm}
  \item \textbf{Potential heat, cold, and soil pH stress tolerance loci were identified at low frequency and are targets for selection}
\end{itemize}
% \end{large}


\end{alertblock}

%----------------------------------------------------------------------------------------
%	INTRODUCTION
%----------------------------------------------------------------------------------------

\begin{block}{Introduction}


%% Text
Abiotic stresses such as heat and frost threaten production of quality cranberry (\textit{Vaccinium macrocarpon} Aiton.) fruit.

\vspace{0.5cm}

Wild cranberry germplasm may contain genetic variation for tolerance to these abiotic stresses.

\vspace{0.5cm}

Identifying genomic regions associated with adaptation to environments would facilitate the targeted use of wild germplasm in breeding programs.

\vspace{0.5cm}

\textbf{Environmental association analysis} correlates allele frequencies in natural populations with environmental factors \cite{Walsh2018}. This can help identify genomic regions potentially underlying adaptive variation to abiotic stresses \cite{Yoder2014, Anderson2016, Lei2019}.

\vspace{0.5cm}

\textbf{Objective:} Use environmental association to identify genomic regions associated with local environmental factors in wild cranberry populations.


\end{block}

\vspace{2cm}


% %----------------------------------------------------------------------------------------
% %	OBJECTIVES OR QUESTIONS
% %----------------------------------------------------------------------------------------
%
% % Set the color
% \setbeamercolor{block alerted title}{fg=white,bg=blueset1!70} % Change the alert block title colors
% \setbeamercolor{block alerted body}{fg=black,bg=blueset1!10} % Change the alert block body colors
%
% \begin{alertblock}{\large{Objective}}
%
% \begin{itemize}
%   \item{\textbf{Identify genomic regions associated with local environmental factors in wild cranberry populations.}}
%   % \item{\textbf{Determine candidate abiotic stress tolerance loci as potential selection targets.}}
%   % \item \textbf{What genomic regions are associated with local environmental conditions in wild cranberry?}
%   % \item What is the frequency and occurrance of putatively adaptive loci in native cranberry selections?
% \end{itemize}
%
%
% \end{alertblock}


%----------------------------------------------------------------------------------------
%	MATERIALS AND MATERIALS
%----------------------------------------------------------------------------------------

\begin{block}{Materials and Methods}

We used a collection of 111 wild cranberry individuals sampled from 17 locations (populations) between 1991 and 2000.


% Figure with sampling locations
\begin{figure}
  \begin{center}
    \includegraphics[width=0.75\linewidth]{figures/figure2.jpg}
  \end{center}
\end{figure}

\vspace{1cm}

4,582 SNP marker genotypes were obtained using genotyping-by-sequencing \cite{Diaz-Garcia2020}.


\vspace{1cm}

We obtained bioclimatic data (19 WorldClim \cite{Hijmans2005} variables) and soil data (20 SoilGrids \cite{Hengl2014} variables) for each cranberry population.


\begin{table}[!h]
\centering
\begin{tabular}{lcl}
\toprule
\textbf{Category} & \textbf{$N_{Variable}$} & \textbf{Example Variables}\\
\midrule
Geography & 3 & Elevation, Latitude, Longitude\\
Temperature & 11 & Annual Mean Temperature, Mean Diurnal
Range\\
Precipitation & 8 & Annual Precipitation, Precipitation of
Wettest Month\\
Soil & 20 & Bulk Density Subsoil, pH Topsoil, \% Sand topsoil\\
\bottomrule
\end{tabular}
\end{table}

\vspace{2cm}

We conducted an environmental association using a mixed-model framework (similar to traditional genomewide association studies):

% \begin{center}
% \begin{Large}
% $ \mathbf{y = Sa + Xb + Zu + e} $
% \end{Large}
% \end{center}

% Figure with sampling locations
\begin{figure}
  \begin{center}
    \includegraphics[width=0.9\linewidth]{figures/eaa_model1.jpg}
  \end{center}
\end{figure}


% End the M&M block
\end{block}



%----------------------------------------------------------------------------------------

\end{column} % End of the first column (i.e. Introduction, objectives, and methods)

%----------------------------------------------------------------------------------------

% Begin the results columns

\begin{column}{\sepwid}\end{column} % Empty spacer column

\begin{column}{\twocolwid} % Begin a column which is two columns wide (column 2)


% Removing the two-columns within the middle column structure
% \begin{columns}[t,totalwidth=\twocolwid] % Split up the two columns wide column
%
% \begin{column}{\onecolwid}\vspace{-.6in} % The first column within column 2 (column 2.1)

%----------------------------------------------------------------------------------------
%	RESULTS
%----------------------------------------------------------------------------------------

\begin{block}{Results}

% Now split into two columns
\begin{columns}[t,totalwidth=\twocolwid] % Split up the two columns wide column


\begin{column}{\onecolwid} % The first column within column 2 (column 2.1)

%----------------------------------------------------------------------------------------

%% Figure of the first example

\vspace{0.5cm}

%% Figure 3a

\begin{center}
\textbf{Example 1: chromosome 4 region associated with multiple temperature variables}
\end{center}

\begin{center}
\begin{figure}
  \includegraphics[width=1\linewidth]{figures/Figure3a.jpg}
\end{figure}
\end{center}

\vspace{2cm}


The \textbf{A} marker allele is associated with colder temperatures and wider temperature ranges.

%% Figure 3b

\begin{center}
\begin{figure}
  \includegraphics[width=1\linewidth]{figures/Figure3b.jpg}
\end{figure}
\end{center}


\vspace{2cm}


This favorable allele is at low frequency among wild cranberry populations.

%% Figure 3c

\begin{center}
\begin{figure}
  \includegraphics[width=1\linewidth]{figures/Figure3c.jpg}
\end{figure}
\end{center}



%----------------------------------------------------------------------------------------

\end{column} % End of column 2.1

\begin{column}{\onecolwid} % The second column within column 2 (column 2.2)

%----------------------------------------------------------------------------------------

%% Figure 4

\vspace{0.5cm}


\begin{center}
\textbf{Example 2: genomic regions associated with soil characteristics}
\end{center}


A low-frequency allele on chromosome 2 is associated with reduced topsoil sand content.


%% Figure 4a
\begin{center}
\begin{figure}
  \includegraphics[width=1\linewidth]{figures/Figure4a.jpg}
\end{figure}
\end{center}

% Add some space
\vspace{1cm}

%% Figure 4b
\begin{center}
\begin{figure}
  \includegraphics[width=1\linewidth]{figures/Figure4b.jpg}
\end{figure}
\end{center}



\vspace{3cm}

A chromosome 3 locus is associated with soil pH; the allele associated with more alkaline soil is at low frequency in multiple cranberry populations.


%% Figure 5a
\begin{center}
\begin{figure}
  \includegraphics[width=1\linewidth]{figures/Figure5a.jpg}
\end{figure}
\end{center}

% Add some space
\vspace{1cm}

%% Figure 5b
\begin{center}
\begin{figure}
  \includegraphics[width=1\linewidth]{figures/Figure5b.jpg}
\end{figure}
\end{center}



%% Add some space
\vspace{2cm}



%% Figure of the second module
\begin{alertblock}{Summary of environmental associations}

\begin{itemize}
  \item{143 marker associations were detected for 36 environmental variables}
  \item{74 unique markers across all 12 chromosomes were associated with bioclimatic variables}
  \item{5 markers were associated with $\ge$ 5 variables}
\end{itemize}

\end{alertblock}


%----------------------------------------------------------------------------------------

\end{column} % End of column 2.2

\end{columns} % End of the split of column 2


%-------------------------------

% % Insert an alert block that spans the whole width
%
% % Set the color
% \setbeamercolor{block alerted title}{fg=white,bg=orangeset1!70} % Change the alert block title colors
% \setbeamercolor{block alerted body}{fg=black,bg=orangeset1!10} % Change the alert block body colors
%
% \begin{alertblock}{}
%
% % Divide into columns
% % Now split into two columns
% \begin{columns}[t,totalwidth=\twocolwid] % Split up the two columns wide column
%
% \begin{column}{\onecolwid} %
%
% %%% Fill in here
%
% \end{column} % End of column
%
% %%-----------------------------------
%
% \begin{column}{\onecolwid} % The second column within column 2 (column 2.2)
%
% %%% Fill in here
%
%
% \end{column} % End of column
%
% \end{columns} % End of the split of column
%
% \end{alertblock}

%--------------------------------



% End the results block
\end{block}

%----------------------------



%% Add some space
\vspace{0cm}



%----------------------------------------------------------------------------------------
% Split into columns
\begin{columns}[t,totalwidth=\twocolwid] % Split up the two columns wide column
%----------------------------------------------------------------------------------------

%----------------------------------------------------------------------------------------
%	DISCUSSION/CONCLUSIONS
%----------------------------------------------------------------------------------------

\begin{column}{1\onecolwid}

% Set the color
\setbeamercolor{block alerted title}{fg=white,bg=orangeset1!70} % Change the alert block title colors
\setbeamercolor{block alerted body}{fg=black,bg=orangeset1!10} % Change the alert block body colors

\begin{alertblock}{Implications for cranberry improvement}

\begin{large}
\begin{itemize}
  \item{We identified genomic regions associated with environmental factors related to important cranberry abiotic stresses.}
  \item{Low-frequency alleles favorably associated with abiotic stresses are ideal targets for selection.}
  \item{Individuals possessing these alleles are candidates for pre-breeding germplasm.}
\end{itemize}
\end{large}



\end{alertblock}


\end{column}



%----------------------------------------------------------------------------------------

%% Open an Acknowledgements/Reference column

\begin{column}{1\onecolwid}



%----------------------------------------------------------------------------------------
%	ACKNOWLEDGEMENTS
%----------------------------------------------------------------------------------------


% Change the font of the block title for the next two sections
% \setbeamerfont{block title}{size=\small}

\begin{block}{\large{Acknowledgements}}

\begin{footnotesize}

Thanks go to Jennifer Johnson-Cicalese for helpful input and to Juan Zelapa and Eric Wiesman for providing the raw genotyping-by-sequencing reads.

\end{footnotesize}

\end{block}


%----------------------------------------------------------------------------------------
%	REFERENCES
%----------------------------------------------------------------------------------------

\begin{block}{\large{References}}

% \nocite{*} % Insert publications even if they are not cited in the poster
% ^ removed this so list only those publications that are cited

% \setbeamercolor{

\begin{tiny}

\bibliographystyle{supporting_files/posterbibstyle}
\bibliography{\mendlibrary}


\end{tiny}

% End the reference block
\end{block}


% %----------------------------------------------------------------------------------------
% %	CONTACT INFORMATION
% %----------------------------------------------------------------------------------------
%
% \setbeamercolor{block alerted title}{fg=white,bg=orangeset1!70} % Change the alert block title colors
% \setbeamercolor{block alerted body}{fg=black,bg=orangeset1!10} % Change the alert block body colors
%
% \begin{alertblock}{\large{\textsuperscript{*}Contact Information}}
%
% \begin{itemize}
% \item Email: \href{mailto:neyha001@umn.edu}{neyha001@umn.edu}
% \item Twitter: \href{https://twitter.com/neyhartje}{@neyhartje}
% \item Web: \href{http://smithlab.cfans.umn.edu/}{smithlab.cfans.umn.edu}
% \end{itemize}
%
% \end{alertblock}


%----------------------------------------------------------------------------------------
% End the acknowlegements/references column
\end{column}
%----------------------------------------------------------------------------------------

%----------------------------------------------------------------------------------------
% End the two bottom columns
\end{columns}
%----------------------------------------------------------------------------------------

%----------------------------------------------------------------------------------------
% End the second main column
\end{column} % End of the third column
%----------------------------------------------------------------------------------------



\end{columns} % End of all the columns in the poster

\end{frame} % End of the enclosing frame

\end{document} % End the poster document
