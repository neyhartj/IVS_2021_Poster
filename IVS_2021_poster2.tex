%%%%%%%%%%%%%%%%%%%%%%%%%%%%%%%%%%%%%%%%%
% Jacobs Landscape Poster
% LaTeX Template
% Version 1.1 (14/06/14)
%
% Created by:
% Computational Physics and Biophysics Group, Jacobs University
% https://teamwork.jacobs-university.de:8443/confluence/display/CoPandBiG/LaTeX+Poster
%
% Further modified by:
% Nathaniel Johnston (nathaniel@njohnston.ca)
%
% This template has been downloaded from:
% http://www.LaTeXTemplates.com
%
% License:
% CC BY-NC-SA 3.0 (http://creativecommons.org/licenses/by-nc-sa/3.0/)
%
%%%%%%%%%%%%%%%%%%%%%%%%%%%%%%%%%%%%%%%%%

%----------------------------------------------------------------------------------------
%	PACKAGES AND OTHER DOCUMENT CONFIGURATIONS
%----------------------------------------------------------------------------------------

\documentclass[final]{beamer}

\usepackage[scale=1.24,orientation=portrait]{beamerposter} % Use the beamerposter package for laying out the poster
% Add orientation=portrait for a portrait mode

\usetheme{confposter} % Use the confposter theme supplied with this template

\usepackage{wrapfig} % Use for wrapping figures

\usepackage[super,numbers]{natbib} % Use for fine-tunning the captions. This is also required for using the poster bibiolography style
% that I created

\usepackage[export]{adjustbox} % Use for alignning images
\usepackage{subcaption}

% \usepackage{biblatex} % Bibliography

\setbeamercolor{block title}{fg=greenset1,bg=white} % Colors of the block titles
\setbeamercolor{block body}{fg=black,bg=white} % Colors of the body of blocks
\setbeamercolor{block alerted title}{fg=white,bg=blueset1!70} % Colors of the highlighted block titles
\setbeamercolor{block alerted body}{fg=black,bg=blueset1!10} % Colors of the body of highlighted blocks

% Many more colors are available for use in beamerthemeconfposter.sty

%-----------------------------------------------------------

% Define the column widths and overall poster size
% To set effective sepwid, onecolwid and twocolwid values, first choose how many columns you want and how much separation you want between columns
% In this template, the separation width chosen is 0.02 of the paper width and a 3-column layout
% onecolwid should therefore be (1-(ncol + 1)*sepwid)/ncol e.g. (1-(3+1)*0.024)/3 = 0.3013333
% Set twocolwid to be (2*onecolwid)+sepwid = 0.6266666
% Set threecolwid to be (3*onecolwid)+2*sepwid = 0.9279999

\newlength{\sepwid}
\newlength{\onecolwid}
\newlength{\twocolwid}
\newlength{\threecolwid}

%% Poster size
\setlength{\paperwidth}{34in} % A0 width: 46.8in
\setlength{\paperheight}{46in} % A0 height: 33.1in


\setlength{\sepwid}{0.024\paperwidth} % Separation width (white space) between columns
\setlength{\onecolwid}{0.3013333\paperwidth} % Width of one column
\setlength{\twocolwid}{0.6266666\paperwidth} % Width of two columns
\setlength{\threecolwid}{0.9279999\paperwidth} % Width of three columns
\setlength{\topmargin}{-0.5in} % Reduce the top margin size
%-----------------------------------------------------------

\usepackage{graphicx}  % Required for including images

\usepackage{booktabs} % Top and bottom rules for tables

%----------------------------------------------------------------------------------------
%	TITLE SECTION
%----------------------------------------------------------------------------------------

\title{Identification of Candidate Abiotic Stress Tolerance Loci in Wild Cranberry Using Environmental Association} % Poster title
% Add \\ to induce new line

\vspace{1cm}

\author{Jeffrey L. Neyhart\textsuperscript{1*}, Michael Kantar\textsuperscript{2}, Joseph Kawash\textsuperscript{1}, James Polashock\textsuperscript{1}, and Nicholi Vorsa\textsuperscript{3}} % Author(s)

\institute{
\textsuperscript{1}USDA-ARS, Genetic Improvement for Fruits and Vegetables Laboratory, \textsuperscript{2}Department of Tropical Plant and Soil Sciences, University of Hawai'i at Mānoa, and \textsuperscript{3}Department of Plant Biology, Rutgers University

\small{*Contact: Email - \href{mailto:jeffrey.neyhart@usda.gov}{jeffrey.neyhart@usda.gov}}

}



%----------------------------------------------------------------------------------------

\begin{document}

\addtobeamertemplate{block end}{}{\vspace*{2ex}} % White space under blocks
\addtobeamertemplate{block alerted end}{}{\vspace*{2ex}} % White space under highlighted (alert) blocks

\setlength{\belowcaptionskip}{2ex} % White space under figures
\setlength\belowdisplayshortskip{2ex} % White space under equations

\begin{frame}[t] % The whole poster is enclosed in one beamer frame

\begin{columns}[t] % The whole poster consists of three major columns, the second of which is split into two columns twice
% the [t] option aligns each column's content to the top

%----------------------------------------------------------------------------------------

\begin{column}{\sepwid}\end{column} % Empty spacer column

\begin{column}{\onecolwid} % The first column


%----------------------------------------------------------------------------------------
%	TAKEAWAYS
%----------------------------------------------------------------------------------------

\setbeamercolor{block alerted title}{fg=white,bg=orangeset1!70} % Change the alert block title colors
\setbeamercolor{block alerted body}{fg=black,bg=orangeset1!10} % Change the alert block body colors

\begin{alertblock}{\Large{Takeaways}}

% \begin{textbf}

\begin{itemize}
  \item \textbf{A multi-environment barley (\textit{Hordeum vulgare} L.) dataset was used to test genomewide predictions}
  \item \textbf{Environments in a training set were clustered based on phenotype data or environmental covariables}
  \item \textbf{Environment clusters may improve prediction accuracy, but using all data is simpler and effective}
\end{itemize}

% \end{textbf}

\end{alertblock}

%----------------------------------------------------------------------------------------
%	INTRODUCTION
%----------------------------------------------------------------------------------------

\begin{block}{Introduction}

% \begin{footnotesize}

Genomewide selection may allow breeders to predict the merit of unphenotyped individuals in future environments \cite{Malosetti2016a, Jarquin2017}.

\vspace{0.5cm}

Predictions for future environments require information on those environments, as from historical phenotypic data or environmental covariables (ECs).

\vspace{0.5cm}

Environmental clusters can reduce genotype-environment interactions (GxE)\cite{Bernardo2010} and potentially improve predictions \cite{Lado2016a}.

\vspace{0.5cm}

Clustering methods have not been thoroughly tested.


% \end{footnotesize}

\end{block}


%----------------------------------------------------------------------------------------
%	OBJECTIVES OR QUESTIONS
%----------------------------------------------------------------------------------------

% Set the color
\setbeamercolor{block alerted title}{fg=white,bg=blueset1!70} % Change the alert block title colors
\setbeamercolor{block alerted body}{fg=black,bg=blueset1!10} % Change the alert block body colors

\begin{alertblock}{\large{Questions}}

\begin{footnotesize}

\begin{itemize}r
  \item What environmental clustering strategies increase genomewide prediction accuracy?
  \item Are environmental clusters superior to using all data when predicting an uphenotyped population?
\end{itemize}

\end{footnotesize}


\end{alertblock}


%----------------------------------------------------------------------------------------
%	MATERIALS AND MATERIALS
%----------------------------------------------------------------------------------------

\begin{block}{Materials and Methods}

Training (TP) and validation populations (VP) were phenotyped for \textit{heading date}, \textit{grain yield}, and \textit{plant height}.


% Figure with traits and number of environments plus the map of the trial locations
\begin{figure}
  \begin{center}
  \begin{minipage}[m]{0.50\linewidth}
    \centering
    \includegraphics[width=\linewidth]{figures/site_map.jpg}
  \end{minipage}% No spacing
  \begin{minipage}[m]{0.50\linewidth}
    \centering
    \includegraphics[width=\linewidth]{figures/populations.png}
  \end{minipage}% No spacing
\end{center}
\end{figure}


\vspace{1cm}

%% Phenotypic analysis
Environment effect ($t_j$) and interaction score ($IPCA_j$) calculated from additive main effects and multiplicate interaction (AMMI) model:
  $$y_{ij} = \mu + g_i + t_j + \sum^N_{n=1} (IPCA_{in})(IPCA_{jn}) + \delta_{ij} + \epsilon_{ij}$$


\vspace{1cm}


Environmental covariables:

\begin{footnotesize}

\begin{itemize}
  \item Covariables related to temperature, daylength, rainfall, and soil properties were selected
  \item Daily observations obtained during growing period
  \item Data was averaged over the 10 years preceding an environment
\end{itemize}

\end{footnotesize}


\vspace{1cm}


Distance measures to cluster environments:

\vspace{-0.5cm}

\begin{scriptsize}
\begin{table}[h]
\centering
\begin{tabular}{p{0.15\linewidth} p{0.58\linewidth} p{0.01\linewidth} p{0.21\linewidth}}
\toprule
\textbf{Distance measure} & \textbf{Definition} & & \textbf{Predict future environment?}\\
\midrule
PD & Phenotypic distance between environments based on genetic correlation \cite{Ouyang1995} & & No \\
LocPD & Phenotypic distance between locations & & Yes, if location was observed\\
GCD & Great circle distance using latitude/longitude & & Yes\\
All-EC & All environmental covariables (ECs) & & Yes\\
Mean-EC & ECs correlated with environment effect & & Yes\\
IPCA-EC & ECs correlated with environment IPCA score & & Yes\\
\bottomrule
\end{tabular}
\end{table}
\end{scriptsize}







% End the M&M block
\end{block}


%----------------------------------------------------------------------------------------

\end{column} % End of the first column (i.e. Introduction, objectives, and methods)

\begin{column}{\sepwid}\end{column} % Empty spacer column

\begin{column}{\twocolwid} % Begin a column which is two columns wide (column 2)


% Removing the two-columns within the middle column structure
% \begin{columns}[t,totalwidth=\twocolwid] % Split up the two columns wide column
%
% \begin{column}{\onecolwid}\vspace{-.6in} % The first column within column 2 (column 2.1)

%----------------------------------------------------------------------------------------
%	RESULTS
%----------------------------------------------------------------------------------------

\begin{block}{Results}

% Now split into two columns
\begin{columns}[t,totalwidth=\twocolwid] % Split up the two columns wide column


\begin{column}{0.6\onecolwid} % The first column within column 2 (column 2.1)

%----------------------------------------------------------------------------------------


% Table with variance components


\begin{footnotesize}

\textbf{GxE variance attributed to lack of correlation between environments.}

\end{footnotesize}



\begin{scriptsize}

\begin{table}[!h]
\centering
\begin{tabular}{p{0.45\linewidth} p{0.1833333\linewidth} p{0.1833333\linewidth} p{0.1833333\linewidth}}
\toprule
\textbf{Source of variation} & \textbf{Grain Yield} & \textbf{Heading Date} & \textbf{Plant Height} \\
\midrule
Genotype & 2.3\% & 21\% & 5\%\\
Environment & 82\% & 68\% & 84\%\\
G x E & 6.8\% & 11\% & 11\%\\
\hspace{0.1cm} $V_G$ Heterogeneity* & \hspace{0.1cm} 9\% & \hspace{0.1cm} 18\% & \hspace{0.1cm} 9.6\%\\
\hspace{0.1cm} Lack Of  Correlation* & \hspace{0.1cm} 91\% & \hspace{0.1cm} 82\% & \hspace{0.1cm} 90\%\\
Residual & 8.7\% & 0.00019\% & 0.0053\%\\
\bottomrule
\multicolumn{4}{l}{\tiny{*Reported as proportion of G x E variance.}}\\
\end{tabular}
\end{table}

\end{scriptsize}






% % Insert two lines
% \begin{tikzpicture}[remember picture,overlay]
%   \shade [inner color=gray,outer color=white]
%   (\textwidth+1cm,0) rectangle (\textwidth+0.3cm+1cm,27);
% \end{tikzpicture}
%
% % Insert two lines
% \begin{tikzpicture}[remember picture,overlay]
%   \shade [inner color=gray,outer color=white]
%   (0,0) rectangle (\textwidth,0.3cm);
% \end{tikzpicture}



%----------------------------------------------------------------------------------------

\end{column} % End of column 2.1

\begin{column}{1.4\onecolwid} % The second column within column 2 (column 2.2)

%----------------------------------------------------------------------------------------

% Figure with environmental correlations

\begin{figure}
  \includegraphics[width=1\linewidth]{figures/environmental_correlation_tp.jpg}
\end{figure}




%----------------------------------------------------------------------------------------

\end{column} % End of column 2.2

\end{columns} % End of the split of column 2


\vspace{4cm}




% Now split into two columns
\begin{columns}[t,totalwidth=\twocolwid] % Split up the two columns wide column

% First column
\begin{column}{1\onecolwid} % The first column within column 2 (column 2.1)


% Title of figure


%% Figure of AMMI results
\begin{figure}
  \includegraphics[width=1\linewidth]{figures/ammi_biplots_complete.jpg}
\end{figure}

\vspace{0.5cm}

\begin{footnotesize}

\textbf{AMMI analysis did not indicated strong clustering of environments by location or year.}

\end{footnotesize}



% End the first column
\end{column} % End of column 2.1


\begin{column}{1\onecolwid} % The second column within column 2 (column 2.2)


\begin{figure}
  \begin{center}
  \begin{minipage}[m]{0.48\linewidth}
    \centering
    \includegraphics[width=1\linewidth]{figures/env_mean_cor_top_poster.jpg}
  \end{minipage}% No spacing
  \hspace{0.02\linewidth}
  \begin{minipage}[m]{0.48\linewidth}
    \centering
    \includegraphics[width=1\linewidth]{figures/env_ipca_cor_top_poster.jpg}
  \end{minipage}% No spacing
\end{center}
\end{figure}


\begin{footnotesize}

\textbf{Informative environmental covariables were correlated with environment effects and IPCA scores.}

\end{footnotesize}



\end{column} % End of column 2.3



\end{columns} % End of the outer split




\vspace{3cm}



%%% Start the third results block (predictions)
% Now split into two columns
\begin{columns}[t,totalwidth=\twocolwid] % Split up the two columns wide column

% First column
\begin{column}{1\onecolwid} % The first column within column 3 (column 3.1)


%% Figure of cluster prediction scheme
\begin{figure}
  \includegraphics[width=1\linewidth]{figures/cluster_prediction_diagram.png}
\end{figure}


\end{column} % End of column 3.1




% First column
\begin{column}{1\onecolwid} % The first column within column 3 (column 3.1)



%% Figure of cluster prediction results
\begin{figure}
  \includegraphics[width=0.90\linewidth]{figures/cluster_model_effect_poster.jpg}
\end{figure}

\vspace{0.5cm}

\begin{footnotesize}

\textbf{Training sets clustered by ECs may increase accuracy, but using all data is generally simpler and comparable.}

\end{footnotesize}




\end{column} % End of column 3.2

\end{columns} % End of the third split





% Add some space
\vspace{1cm}


% End the first results block
\end{block}

%----------------------------------------------------------------------------------------



% Split into three columns - one for each of these sections
\begin{columns}[t,totalwidth=\twocolwid] % Split up the two columns wide column

%----------------------------------------------------------------------------------------
%	CONCLUSIONS
%----------------------------------------------------------------------------------------

\begin{column}{0.6667\onecolwid}

% Set the color
\setbeamercolor{block alerted title}{fg=white,bg=orangeset1!70} % Change the alert block title colors
\setbeamercolor{block alerted body}{fg=black,bg=orangeset1!10} % Change the alert block body colors

\begin{alertblock}{Conclusions}

\begin{footnotesize}

% % Body
% Genomewide selection can be used to predict untested genotypes in future environments.
Training sets of clustered environments can improve prediction accuracy. Clusters based on phenotypic data tend to be superior.

\vspace{0.5cm}

Environmental covariates can be used to predict future environments.

\vspace{0.5cm}

Using all available training data to predict untested genotypes in future environments is simpler and equally effective.



\end{footnotesize}

\end{alertblock}


\end{column}



%----------------------------------------------------------------------------------------
%	ACKNOWLEDGEMENTS
%----------------------------------------------------------------------------------------

\begin{column}{0.6667\onecolwid} % The first column within column 2 (column 2.1)


% Change the font of the block title for the next two sections
% \setbeamerfont{block title}{size=\small}

\begin{block}{\large{Acknowledgements}}

\begin{tiny}

We thank the many collaborators that made this project successful; E. Schiefelbein, G. Velasquez, and K. Beaubien for technical support; and members the Smith Lab for assistance, advice, and encouragement. This research is supported by the U.S. Wheat and Barley Scab Initiative, the Minnesota Department of Agriculture, Rahr Malting Co., the Brewers Association, and USDA-NIFA Grant No. 2018-67011-28075. Any opinions, findings, conclusions, or recommendations expressed in this publication are those of the author(s) and do not necessarily reflect the view of the U.S. Department of Agriculture.

\vspace{0.25cm}

% Path for the funding logos
\graphicspath{{C:/Users/jln54/GoogleDrive/BarleyLab/ForKevin/Resources/}}


% Funding logos
\begin{center}
% \includegraphics[height=4cm]{figures/umn_logo.png}
% \hspace{.5cm}
\includegraphics[height=1.5cm]{uswbsi_logo.png}
\hspace{.0cm}
\includegraphics[height=1.5cm]{MDA-logo.jpg}
\hspace{.0cm}
\includegraphics[height=1.5cm]{rahr_logo.png}
\hspace{.0cm}
\includegraphics[height=1.5cm]{brewersassociation_logo.png}
\hspace{0.0cm}
\includegraphics[height=1.5cm]{usda_nifa_horizontal_cmyk_300.jpg}
\end{center}

\end{tiny}

\end{block}

% End the acknowledgements column
\end{column}



%----------------
% Begin the references column
\begin{column}{0.6667\onecolwid} % The first column within column 2 (column 2.1)

%----------------------------------------------------------------------------------------
%	REFERENCES
%----------------------------------------------------------------------------------------

\begin{block}{\large{References}}

% \nocite{*} % Insert publications even if they are not cited in the poster
% ^ removed this so list only those publications that are cited

% \setbeamercolor{

\begin{tiny}

\bibliographystyle{C:/Users/jln54/GoogleDrive/Literature/BibliographyStyles/posterbibstyle}
\bibliography{C:/Users/jln54/GoogleDrive/Literature/MendeleyLibrary/library}


\end{tiny}

\end{block}

% End the references column
\end{column}

% End the two-column format
\end{columns}

% %----------------------------------------------------------------------------------------
% %	CONTACT INFORMATION
% %----------------------------------------------------------------------------------------
%
% \setbeamercolor{block alerted title}{fg=white,bg=orangeset1!70} % Change the alert block title colors
% \setbeamercolor{block alerted body}{fg=black,bg=orangeset1!10} % Change the alert block body colors
%
% \begin{alertblock}{\large{\textsuperscript{*}Contact Information}}
%
% \begin{itemize}
% \item Email: \href{mailto:neyha001@umn.edu}{neyha001@umn.edu}
% \item Twitter: \href{https://twitter.com/neyhartje}{@neyhartje}
% \item Web: \href{http://smithlab.cfans.umn.edu/}{smithlab.cfans.umn.edu}
% \end{itemize}
%
% \end{alertblock}


%----------------------------------------------------------------------------------------

\end{column} % End of the third column

\end{columns} % End of all the columns in the poster

\end{frame} % End of the enclosing frame

\end{document}
